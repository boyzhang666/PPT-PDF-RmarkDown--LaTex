\usepackage{ctex}
\setmainfont{Times New Roman} % 设置英文字体
%\setCJKmainfont{宋体}		% 设置中文字体
\usepackage{amsmath, amsfonts, amssymb, amsthm}                                   %\使用宏包{美国数学协会数学}


%===============================设置页脚格式===============================
\setbeamertemplate{footline}{%
	\leavevmode%
	\hbox{%
		\begin{beamercolorbox}[wd=0.3\paperwidth,ht=2.25ex,dp=1ex,center]{author in head/foot}%
			\usebeamerfont{author in head/foot}\insertshortauthor \quad (\insertshortinstitute)
		\end{beamercolorbox}%
		\begin{beamercolorbox}[wd=0.4\paperwidth,ht=2.25ex,dp=1ex,center]{title in head/foot}%
			\usebeamerfont{title in head/foot}\insertshorttitle
		\end{beamercolorbox}%
		\begin{beamercolorbox}[wd=0.3\paperwidth,ht=2.25ex,dp=1ex,center]{date in head/foot}%
			\usebeamerfont{date in head/foot}\insertshortdate{} \quad
			\insertframenumber{} / \inserttotalframenumber
	\end{beamercolorbox}}%
	\vskip0pt%
}


%===============================设置主题样式和主题色===============================
\usetheme{AnnArbor}                                                 %全局主题
%\usetheme{Berlin} %不显示底栏
%\usetheme{cambridgeUS}
%\usetheme{Darmstadt}
%\usetheme{Frankfurt}
%\usetheme{Ilmenau}
%\usetheme{Hannover}
%\usetheme{Berkeley}
%\usetheme{EastLansing}

\useinnertheme{circles}                                              %内部主题                                        
%内部主题,内部主题主要控制的是标题页,列表项目、定理环境、图表环境、脚注在一帧内的内容格式(做幻灯片的时候,其实一般不建议使用脚注)。预定义的内部主题格式有:default、circles、rounded、rectangles等。

\useoutertheme{miniframes}                                           %外部主题,常用:default、miniframes、tree、smoothtree
%外部主题,\useoutertheme外部主题主要控制的是幻灯片顶部尾部的信息栏、边栏、图标、帧标题等一帧之外的格式。预定义的外部主题有default、infolines、miniframes、tree、smoothtree等。

\usecolortheme{spruce}                                               %色彩主题,该色调很多不显示底栏
%色彩主题,\usecolortheme色彩主题控制各个部分的色彩。预定义的色彩主题包括default、albatross、beaver、beetle、crane、dolphine、dove、fly、lily、orchid、rose、seagull、seahorse、sidebartab、structure、whale、wolverine等。

\usefonttheme{serif}                                                 %字体主题
%字体主题,\usefonttheme字体主题则控制幻灯片的整体字体风格。预定义的beamer字体主要包括default、professionalfonts、serif、structurebold、structureitalicserif、structuresmallcapsserif等。其中默认字体主题default的效果是整个幻灯片使用无衬线字体,这是多数幻灯片的选择;serif主题则是衬线字体,不过此时最好使用较大的字号和较粗的字体;professionalfonts不对字体有特别的设置,需要使用另外的专门的宏包进行设置;structure开头的几个主题则对beamer中的几个结构有特别设置。


% ===============================全局模式下的各项的色彩设置===============================




\setbeamercolor{author}{fg=black}  %首页作者字体颜色
\setbeamercolor{date}{fg=black}     %首页时间字体颜色

\setbeamertemplate{itemize items}{\color{zdyred}$\bullet$}                                     %项目编号
\setbeamercolor{section number projected}{bg=gray!20!,fg=zdyblue}                       %目录圆圈编号

\setbeamercolor{titlelike}{bg=gray!35!,fg=zdyblue}                                                     %首页标题页的标题 设置底框颜色及字体颜色
\setbeamercolor{frametitle}{bg=white,fg=zdyblue}                                                     %每页的 标题的底框和字体颜色设置

\setbeamercolor{palette secondary}{bg=gray!50!,fg=zdyblue}                                 % 底部左边条带和顶部第二个条带的字体和背景颜色  
\setbeamercolor{palette tertiary}{bg=zdyblue,fg=white}                                           %底部中间条带和顶部第一个条带的字体和背景颜色
\setbeamercolor{palette primary}{bg=gray!50!,fg=zdyblue}                                    %底部右边条带的字体和背景颜色


%\setbeamercolor{palette quaternary}{bg=gray!10!,fg=darkblue}                         %暂时不知道设置谁

%\definecolor{zdyblue}{RGB}{19,63,127}                   
\definecolor{zdyblue}{rgb}{0.0, 0.18, 0.33} 
%\definecolor{orange}{rgb}{1,0.5,0}                          % 1.rgb
%\definecolor{orange}{RGB}{255,127,0}                 % 2.RGB
%\definecolor{zdyred}{HTML}{A80000}                 % 3.HTML
%\definecolor{orange}{cmyk}{0,0.5,1,0}                  % 4.cmyk




%===============================设置logo===============================
%\pgfdeclareimage[height=1.0cm]{university-logo}{otherloads/logo/logoxjublue}         %若需要,只需将学校的LOGO放到同一文件夹下即可
%\logo{\vspace{9.5cm}\pgfuseimage{university-logo}}                   %\logo{\pgfuseimage{university-logo}} 为在右下角

%===============================设置背景===============================
%\setbeamertemplate{background}{\includegraphics[height=\paperheight]{otherloads/background/distinctbgxjublue}}    %设置背景,文件夹内需要有bg.jpg图片


%===============================设置用Acrobat打开就会全屏显示===============================
\hypersetup{pdfpagemode=FullScreen}


